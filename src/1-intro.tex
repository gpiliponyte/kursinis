\sectionnonum{Įvadas}

Priklausomybių valdymo sistemos - tai įrankiai, palengvinantys išorinių modulių naudojimą kuriamose programų
sistemose. Šios sistemos yra ypač svarbios šiomis dienomis, sparčiai augant atvirojo kodo judėjimui, o su šiuo
judėjimu - ir kitų sukurtų modulių naudojimas savo vystomose sistemose. Šiuo metu yra sukurta dešimtys priklausomybių
valdymo įrankių, dažniausiai skirtų išskirtinai vienos programavimo kalbos vartotojams.

Iki šiuolaikinių priklausomybių valdymo sistemų atsiradimo vartotojai buvo patys
atsakingi už reikiamų tiesioginių bei netiesioginių (tranzityvių) priklausomybių paiešką, atsisiuntimą bei jų atnaujinimą. Toks priklausomybių
gavimo procesas buvo ilgas ir nepatogus, taip pat dažnai kildavo klaidos dėl nesuderinamų priklausomybių versijų ar trūkstamų tranzityvių
priklausomybių - šio proceso keliamos problemos lėmė mažesnį išorinių modulių naudojimą programų sistemose. Šiuolaikinės priklausomybių
valdymo sistemos pakeitė negatyvų vartotojų požiūrį į išorinių modulių naudojimą, leisdamos visas reikalingas priklausomybes gauti iš vieno
šaltinio bei automatiškai atsiųsdamos reikiamas tranzityvias priklausomybes. Šios sistemos taip pat suteikė mechanizmus priklausomybių versijų
parinkimui bei priklausomybių atnaujinimui, kurie dar labiau palengvino priklausomybių valdymo procesą. Priklausomybių valdymo sistemos padarė
išorinių modulių naudojimą lengviau prieinamą vartotojams bei leido priklausomybių pagalba sutrumpti programavimo laiką bei susitelti tik
į unikalius vystomos programų sistemos uždavinius.

Priklausomybių valdymo sistemos viena nuo kitos skiriasi ne tik programavimo kalbomis, kuriose yra naudojamos. Tarp
įvairių sistemų naudojami skirtingi algoritmai atlikti tas pačias priklausomybių valdymo funkcijas, tokias kaip
priklausomybių versijų pasirinkimas ar stabilių priklausomybių versijų išlaikymas skirtingose aplinkose.

Šio kursinio darbo tikslas - išanalizuoti rinkoje paplitusias priklausomybių valdymo sistemas ir nustatyti kaip jos įgyvendina pagrindines
priklausomybių valdymo funkcijas. Siekiant šio tikslo išsikelti tokie uždaviniai:
\begin{itemize}
    \item parinkti analizei aktualias priklausomybių valdymo sistemas;
    \item apžvelgti ir išanalizuoti svarbiausias, kaip pasirinktos sistemos realizuoja svarbiausias priklausomybių valdymo sistemų funkcijas;
    \item apibendrinti ir palyginti analizuotų sistemų panašumus ir skirtumus.
\end{itemize}
