\subsection{Nagrinėti pasirinktos priklausomybių sistemos}

Šiuo metu yra sukurta dešimtys skirtingų priklausomybių valdymo sistemų. Tolesniuose skyriuose bus nagrinėjamos
Go, NPM bei Maven sistemos. Šios sistemos pasirinktos dėl kelių priežasčių. NPM turi didžiausią priklausomybių
registrą iš visų priklausomybių valdymo sistemų ir yra vienas dominuojančių įrankių front-end sistemose. Tuo tarpu
Go priklausomybių valdymo mechanizmas įdomus sistemos pertvarkymo metu priimtais unikaliais, iki šiol nenaudotais
priklausomybių valdymo problemų sprendimais. Maven sistema yra šiek tiek senesnė nei Go ar NPM (išleista 2002 metais),
tačiau siūlo daug mechanizmų efektyviau valdyti priklausomybes \cite{PAD17}. Kiekviena iš pasirinktų sistemų turi savitų priklausomybių
valdymo metodų, pavyzdžiui, visos trys iš šių sistemų turi skirtingus algoritmus tranzityvių priklausomybių versijoms gauti.
Svarbus faktorius renkantis sistemas buvo ir tai, jog kiekviena iš minėtų sistemų yra aktyviai naudojama ir šiomis dienomis.
Taigi, visos trys sistemos yra vis dar aktualios ir turi savų ypatumų, leidžiančių juos palyginti kursinio darbo eigoje.