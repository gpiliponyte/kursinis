\subsection{Darbe lyginamos pasirinktų sistemų savybės}

Darbe aptariamos priklausomybių valdymo sistemos (Go, NPM, Maven) bus lyginamos pagal šias priklausomybių valdymo savybes:
\begin{itemize}
    \item priklausomybių versijų parinkimo algoritmus;
    \item priklausomybių versijų stabilumo užtikrinimo mechanizmus;
    \item priklausomybių versijavimo taisykles.
\end{itemize}

Pirmoji savybė, priklausomybių versijų parinkimas, pasirinkta dėl savo kritiškumo priklausomybių valdymo procese, suprantant, jog neteisingas priklausomybių versijų pasirinkimas gali sukelti vystomos programų sistemos trikius ar net visišką neveikimą. Ši savybė taip pat įdomi, nes skirtingi priklausomybių versijų parinkimo algoritmai gali teikti pirmumą skirtingiems priklausomybių valdymo aspektams, tokiems kaip naudojamų priklausomybių naujumas ar stabilumas - šios savybės analizės metu atsispindės ir nagrinėjamos sistemos prioritetai.

Priklausomybių versijų stabilumo užtikrinimas yra dar viena kritinė priklausomybių valdymo funkcija. Ši savybė reikalinga užtikrinti programų sistemos stabilumą skirtingose aplinkose - sistemos produkcinę aplinką turi pasiekti būtent tų versijų priklausomybės, su kuriomis buvo vystoma ir testuojama sistema, kitu atveju gali įvykti netikėtos klaidos. Priklausomybių stabilumo užtikrinimas tampa vis svarbesnis, nes vis daugiau vystytojų naudoja automatinius kodo integravimo ir diegimo įrankius.

Paskutinė savybė, priklausomybių versijavimas, yra pagrindinis būdas priklausomybės autoriui komunikuoti su vartotoju, suteikti informaciją apie tokius priklausomybės aspektus kaip stabilumas bei pokyčiai nuo prieš tai buvusių versijų. Priklausomybių versijavimo savybė nagrinėti pasirinkta norint palyginti, kaip ir kokią informaciją nagrinėtos sistemos perteikia per versijų numerius bei kaip jie standartizuojami.


