\section{Priklausomybių valdymo sistemos}

\subsection{Priklausomybių valdymo sistemų poreikis}

Programų sistemos dažnai susideda iš mažesnių, vieną paskirtį turinčių (ang. single purpose) modulių.
Populiarėjant atvirojo programinio kodo (ang. open source) naudojimui, sistemos neretai priklauso nuo išorinių,
kitų autorių sukurtų modulių \cite{PAD17}. Toks programinio kodo daugkartinis panaudojimas turi daug privalumų - paspartinamas programų
sistemų kūrimo procesas, daugiau laiko skiriama fokusuojantis į unikalias kuriamos sistemos užduotis.

Priklausomybių valdymo sistemos yra įrankiai, palengvinantys išorinių modulių naudojimą programų sistemose.
Šios sistemos leidžia atsisiųsti tiesiogines projekto priklausomybes, dauguma jų taip pat turi mechanizmus gauti
ir projekto tranzityvias priklausomybes (tiesioginių priklausomybių priklausomybes) \cite{PAD17}. Naudojant
priklausomybių valdymo sistemas vartotojui būtina nurodyti tik norimų tiesioginių priklausomybių sąrašą,
tinkamos tranzityvių priklausomybių versijos bei iš kokių repozitorijos jos bus siunčiamos nustatoma
automatiškai. Priklausomybių valdymo sistemos taip pat dažnai suteikia galimybę patogiai atnaujinti individualias
priklausomybes arba visą projekto priklausomybių medį \cite{MAT17}.

Priklausomybių valdymo sistemos yra svarbios kuriant programų sistemas komandoje bei kuriamos programų
sistemos nuolatinės integracijos bei nuolatinio tiekimo (ang. continuous integration and continuous delivery,
trump. CI/CD) procesui. Modernios priklausomybių valdymo sistemos leidžia užtikrinti, jog visi komandos
inžinieriai dirba su tų pačių versijų priklausomybėmis. Šios sistemos taip pat išlaiko vienodas priklausomybes
tiek vystymo, tiek testinėse bei produkcinėje aplinkose \cite{MAT17}. Taip sumažinama su netinkamomis
priklausomybėmis susijusių programų sistemos trikių tikimybė.



\subsection{Priklausomybių valdymo sistemų evoliucija}

\subsection{Darbe lyginamos pasirinktų sistemų savybės}

Darbe aptariamos priklausomybių valdymo sistemos (Go, NPM, Maven) bus lyginamos pagal šias priklausomybių valdymo savybes:
\begin{itemize}
    \item priklausomybių versijų parinkimo algoritmus;
    \item priklausomybių versijų stabilumo užtikrinimo mechanizmus;
    \item priklausomybių versijavimo taisykles.
\end{itemize}

Pirmoji savybė, priklausomybių versijų parinkimas, pasirinkta dėl savo kritiškumo priklausomybių valdymo procese, suprantant, jog neteisingas priklausomybių versijų pasirinkimas gali sukelti vystomos programų sistemos trikius ar net visišką neveikimą. Ši savybė taip pat įdomi, nes skirtingi priklausomybių versijų parinkimo algoritmai gali teikti pirmumą skirtingiems priklausomybių valdymo aspektams, tokiems kaip naudojamų priklausomybių naujumas ar stabilumas - šios savybės analizės metu atsispindės ir nagrinėjamos sistemos prioritetai.

Priklausomybių versijų stabilumo užtikrinimas yra dar viena kritinė priklausomybių valdymo funkcija. Ši savybė reikalinga užtikrinti programų sistemos stabilumą skirtingose aplinkose - sistemos produkcinę aplinką turi pasiekti būtent tų versijų priklausomybės, su kuriomis buvo vystoma ir testuojama sistema, kitu atveju gali įvykti netikėtos klaidos. Priklausomybių stabilumo užtikrinimas tampa vis svarbesnis, nes vis daugiau vystytojų naudoja automatinius kodo integravimo ir diegimo įrankius.

Paskutinė savybė, priklausomybių versijavimas, yra pagrindinis būdas priklausomybės autoriui komunikuoti su vartotoju, suteikti informaciją apie tokius priklausomybės aspektus kaip stabilumas bei pokyčiai nuo prieš tai buvusių versijų. Priklausomybių versijavimo savybė nagrinėti pasirinkta norint palyginti, kaip ir kokią informaciją nagrinėtos sistemos perteikia per versijų numerius bei kaip jie standartizuojami.




\section{Go kalbos priklausomybių valdymo sistema}

Go programavimo kalboje (dar žinomoje kaip Golang), numatytasis priklausomybių valdymo įrankis yra „go get“
komanda. Naudodamasis šia komanda, vartotojas gali atsisiųsti vystomam projektui reikalingas priklausomybes.
Ateinančiuose skyriuose aptariami naujausi „go get“ pokyčiai ir šių pokyčių atnešami privalumai.

\subsection{Priklausomybių valdymo Go kalboje istorija}

Nuo pat Go kalbos išleidimo 2009 metais, jos naudotojų tarpe atsirado poreikis dalintis savo, bei naudoti kitų vartotojų sukurtus paketus.
Šiam tikslui Go inžinieriai sukūrė „GOINSTALL“ komandą, leidžiančią atsisiųsti norimus paketus iš tokių programinio kodo saugyklų kaip Github ar Bitbucket.
Neilgai trukus „GOINSTALL“ pakeitė “go get“ komanda, tačiau abi šios komandos turėjo didžiulį trūkumą - jose nebuvo
paketo versijos sąvokos. Tai reiškė, jog naudojant „go get“ vartotojas visada gaus naujausią šio paketo kopiją \textsuperscript{[COX18a]}.

Negalėjimas pasirinkti paketo versijos kelia dvi pagrindines problemas.
Pirmoji problema yra negalėjimas užtikrinti stabilaus programos surinkimo (ang. stable build),
antroji - nėra galimybių užtikrinti, kad pokyčiai naujoje paketo versijoje bus atgaliai suderinami (ang. backwards-compatible).

Go priklausomybių valdymo trūkumai bandyti spręsti kopijuojant atsiųstus paketus ir laikant juos lokaliai -
šį procesą automatizuoti sukurta daug įrankių, tokių kaip goven, godep, gb. Lokalus paketų laikymas
išsprendė tik stabilaus programos surinkimo (ang. stable build) problemą. Priklausomybių versijų
atgalinio suderinamumo problema vis dar nebuvo išspręsta \textsuperscript{[COX18a]}.

Go kalboje buvo poreikis įvesti tiesioginį paketų versijų valdymą pačioje „go get“ komandoje ir nebepasikliauti trečiųjų šalių
įrankiais. Taip Go inžinieriai pradėjo dirbti prie Vgo - pirmojo apie priklausomybių versijas žinančios (ang. version-aware) Go
komandos prototipo \textsuperscript{[COX18b]}. Go kalbos 1.11 versijoje pradėtas pirminis Vgo prototipe pristatytų idėjų palaikymas
\textsuperscript{[GOLANG19]}.


\subsection{Priklausomybių versijavimas Go sistemoje}

Go inžinierių pasiūlymas dėl version-aware Go komandos implementacijos apsprendžia, kaip naujoje
Go bus versijuojamos priklausomybės. Trečiasis šio pasiūlymo punktas nurodo, kad atnaujintoje Go bus
naudojamas semantinis importų versijavimas (ang. semantic import versioning) \textsuperscript{[COX18c]}.
Semantiniu importų versijavimu siekiama, jog su kiekvienam atgaliai nesuderinamui paketo pakeitimui bus priskiriamas
skirtingas importavimo kelias (ang. import path) su specifikuota pagrindine (ang. major) versija, pavyzdžiui, “github.com/greta/foo/v2".

\begin{figure}[H]
    \centering
    \includegraphics[width=\textwidth]{semantic_import_versioning}
    \caption{Semantinis importų versijavimas \textsuperscript{[COX18d]}}
\end{figure}

Ketvirtasis Go pasiūlymo punktas, importų suderinamumo taisyklė, papildo prieš
tai pasiūlyme pristatytą semantinio importų versijavimo idėją. Ši taisyklė teigia: jei
senas paketas ir naujas paketas turi tą patį importavimo kelią, tuomet naujas paketas privalo būti atgaliai
suderinamas su senuoju \textsuperscript{[COX18c]}.Importų suderinamumo taisyklė paketų autoriams nustato griežtas ribas, kokie pakeitimai leidžiami
nekeičiant paketo importavimo kelio ir kokius pakeitimus įvykdžius būtina kurti naują importavimo kelią.

Naudojant semantinį paketų versijavimą bei laikantis importų suderinamumo taisyklės tikimasi išspręsti prieš tai „go get“
komandoje buvusią nestabilaus API problemą - paketų naudotojams suteikiama garantija,
kad atnaujinant priklausomybes jų naudojamų paketų metodai nesikeis.


\subsection{Priklausomybių versijų pasirinkimas Go sistemoje}

Nuo pat „go get“ pristatymo, viena didžiausių šios komandos problemų buvo
nežinojimas apie valdomų paketų versijas.
Senoji „go get“ komanda turėjo du priklausomybių versijų pasirinkimo algoritmus.
Pirmasis, Go numatytasis algoritmas, „go get B“ metu atsiųsdavo naujausią paketo B versiją
bei naujausias B priklausomybes, kurių nebuvo turima lokaliai. Antrasis algoritmas įvykdžius „go get -u B“
atsiųsdavo naujausią B, bei visas naujausias jos tranzityvių priklausomybių versijas \textsuperscript{[COX18d]}.

Abu šie algoritmai netenkino vartotojų bei kėlė daug klaidų. Naudojant pirmąjį algoritmą,
kilo grėsmė, jog lokaliai turimos priklausomybės bus per senos ir neveiks su naujai atsiųstomis
priklausomybėmis. Antrasis algoritmas taip pat nebuvo visiškai saugus, nes buvo galimybė,
jog naujausios priklausomybių versijos nebus tarpusavyje sutapatinamos (ang. compatible) \textsuperscript{[COX18e]}.

%\begin{figure}[H]
%    \centering
%    \includegraphics[width=\textwidth]{old_go_get}
%    \caption{Problemos „go get“ komandoje \textsuperscript{[COX18e]}}
%\end{figure}

Suprasdami „go get“ priklausomybių versijų pasirinkimo algoritmų keliamas problemas,
Go inžinieriai į pasiūlymą dėl version-aware Go komandos įtraukė ir naują algoritmą priklausomybių
versijų pasirinkimui. Šis algoritmas vadinasi „minimal version selection“ ir siūlo lyg šiol mažai naudotą
priklausomybių versijų pasirinkimo mechanizmą - pasirinkti seniausią leidžiamą paketo versiją.
Dauguma šiuolaikinių priklausomybių valdymo sistemų, tokių kaip dep ar cargo, naudoja priešingą algoritmą -
renkasi naujausią leidžiamą priklausomybės versiją \textsuperscript{[COX18a, COX18f]}.

Russ Cox, vienas pagrindinių Go kūrėjų, teigia, cargo bei dep naudojamas algoritmas yra klaidingas
dėl dviejų priežasčių. Pirmoji priežastis yra tai, jog naujausia leidžiama versija gali nuolat kisti
bei būti nestabili, antroji - klaidos atveju vartotojui reikia skirti papildomo
laiko uždrausti naudoti specifinių versijų priklausomybes \textsuperscript{[COX18a]}.

%\begin{figure}[H]
%    \centering
%    \includegraphics[width=\textwidth]{dep_working}
%    \caption{Naujausios leidžiamos versijos algoritmas dep sistemoje \textsuperscript{[COX18e]}}
%\end{figure}

Go inžinierių pasirinktas „minimal version selection“ algoritmas turi kelis pranašumus.
Šis algoritmas užtikrina, jog visada su ta pačia „go get“ komanda bus gaunamos tų pačių versijų priklausomybes.
Garantija, jog projekto priklausomybės nesikeis, leidžia užtikrinti, jog programos surinkimo rezultatas visada bus toks pats,
tiek programų sistemos kūrimo metu, tiek sistemos produkcinėje aplinkoje \textsuperscript{[COX18a]}. „Minimal version selection“ taip pat
leidžia apsisaugoti nuo naujausiose paketų versijose galinčių būti klaidų - jei paketo A naujausiose versijoje yra klaida,
tiek A paketo autorius, tiek kitų paketų, naudojančių A, autoriai turi laiko ištaisyti klaidą bei
uždrausti naudoti šią trikį turinčią versiją \textsuperscript{[COX18e]}.

%\begin{figure}[H]
%    \centering
%    \includegraphics[width=\textwidth]{vgo_min_version}
%    \caption{„Minimal version selection“ \textsuperscript{[COX18e]}}
%\end{figure}


\section{NPM priklausomybių valdymo sistema}

NPM yra priklausomybių valdymo sistema, naudojama JavaScript aplikacijose.
Ši sistema turi didžiausią priklausomybių registrą pasaulyje, ja naudojantis galima atsisiųsti paketus arba Node modulius.\cite{NPMa}.

\subsection{Priklausomybių versijavimas NPM sistemoje}

NPM priklausomybėms versijuoti naudojama semantinio versijavimo sistema. Ši sistema versijuojamui vienetui
suteikia X.Y.Z formos versiją, kurioje X reiškia pagrindinę (ang. major) versiją, Y - antraeilę (ang. minor) versiją,
Z - pataisymų (ang. patch) versiją. Pataisymų (ang. patch) ir antraeilių (ang. minor) versijų pasikeitimai yra atgaliai suderinami
ir yra saugūs naudoti projektuose su ta pačia pagrindine (ang. major) versija \textsuperscript{[NPMb]}.
Tuo tarpu pagrindinės versijos pasikeitimas reiškia atgaliai nesuderinamus pokyčius.
Semantinis versijavimas leidžia vartotojui susidaryti lūkesčius naujoms priklausomybių versijoms bei
package.json faile apibrėžti kurios priklausomybių versijos ir kurie jų atnaujinimai yra leidžiami projekte.
Ši versijavimo sistema išsprendžia nestabilaus API problemą. %elaborate gurl


\subsection{Priklausomybių versijų pasirinkimas NPM sistemoje}

Norint naudotis NPM priklausomybėmis, projekte būtina turėti package.json failą. Šį failą
sudaro projekto meta-data, tarp kurios nurodomos ir reikiamos priklausomybės bei jų versijos. Package.json
faile išskirtos dvi priklausomybių grupės - \enquote{dependencies} (projekto priklausomybės) bei \enquote{devDependencies}
(projekto kūrimui ir testavimui reikalingos priklausomybės). Kiekvienai šiame faile nurodytai priklausomybei nurodoma ir jos versija \textsuperscript{[GAU18]}.
NPM leidžia nurodyti ne tik tikslias priklausomybių versijas, bet ir leidžiamų versijų diapazoną.
Tildės ženklas (~) leidžia naujesnes pataisymų (ang. patch) versijas,
stogelio ženklas (\textasciicircum) leidžia naujesnes antraeiles (ang. minor) arba pataisymų (ang. patch) versijas \textsuperscript{[PIT15]}.


Galimybė package.json faile esančioms priklausomybėms suteikti leistinų versijų diapazoną turi privalumų ir trūkumų.
Šis metodas leidžia vartotojui patogiau atnaujinti turimas priklausomybes - įvykdžius
\enquote{npm update} komandą automatiškai gaunamos naujausios leistinos priklausomybių versijos vartotojui net
nekeitus package.json \textsuperscript{[PIT15]}. Šio metodo trūkumas yra, jog nurodant leistinų priklausomybių versijų ribas,
dirbant komandoje nėra galimybės užtikrinti, jog įvykdžius \enquote{npm install} su tuo pačiu package.json bus gautos
tų pačių versijų priklausomybės \textsuperscript{[PIT15]}. Išlaikyti stabilias priklausomybių versijas projekte yra svarbu norint
užtikrinti stabilų programos surinkimą (ang. stable build) - jei naujausioje priklausomybės versijoje būtų
klaida, projektas veiktų klaidingai arba visai neveiktų.

Norint išspręsti nestabilaus programos surinkimo problemą, NPM 5 versijoje pradedamas naudoti
package-lock.json failas. Šis failas nurodo tikslų priklausomybių medį - visų naudojamų priklausomybių,
bei šių priklausomybių priklausomybių tikslias versijas \textsuperscript{[NPMc]}. Package-lock.json generuojamas automatiškai pakeitus priklausomybių medį
arba package.json failą \textsuperscript{[NPMc]}. Atsiradus package-lock.json failui, NPM sistema vadovaujasi šiuo failu siųsdama priklausomybes.
Kadangi package-lock.json įrašomos tikslios priklausomybių versijos, išsprendžiama problema,
jog kartu dirbantiems kolegoms ar CI/CD serveriui atsiunčiamos netinkamų versijų priklausomybės.


\section{Maven priklausomybių valdymo sistema}

Apache Maven yra daugiausiai Java projektams naudojama priklausomybių valdymo sistema \cite{MAVENa}.
Naudojant Maven projekte būtinas pom.xml failas, kuriame nurodomos norimos priklausomybės ir jų versijos.
Maven priklausomybės yra vadinamos artefaktais - tai dažniausiai .jar tipo failai, įkelti į Maven repozitoriją.
Pom.xml norimi artefaktai apibūdinami nurodant artefakto groupId, artifactId, version, kartu vadinamus artefakto
koordinatėmis \cite{MAVENb}. Maven Central repozitorija yra numatytoji vieta ieškoti
priklausomybėms, tačiau pom.xml galima pridėti ir kitų repozitorijų (pavyzdžiui,
kompanijų privačias repozitorijas), iš kurių bus siunčiamos priklausomybės \cite{MAVENc}.

Maven taip pat palaiko daugiamodulininius (ang. multi-module) projektus, turinčius tėvinį
modulį su šakniniu (ang. root) pom.xml, bei vaikinius modulius su savo pom.xml paveldinčiais priklausomybes
iš tėvinio modulio \cite{MAVENb}. Tai palengvina pom.xml failų palaikymą.

\subsection{Tranzityvių priklausomybių valdymas Maven sistemoje}

Vartotojo patogumui pom.xml faile būtina nurodyti tik tiesiogines jo projekto priklausomybes -
įrašant reikalingas tiesiogines priklausomybes automatiškai įrašomos ir jų priklausomybės, kitaip žinomos
kaip tranzityvios (ang. transitive) priklausomybės \cite{MAVENd}. Maven turi daug ypatybių leidžiančių patogiau valdyti
projekto tranzityvias priklausomybes.

\subsubsection{Priklausomybių mediacija}
Priklausomybių mediacija (ang. dependency mediation) yra vienas iš Maven siūlomų
funkcionalumų valdyti projekto tranzityvias priklausomybes. Tai algoritmas, nustatantis, kuri
tranzityvios priklausomybės versija turi būti atsiųsta, jeigu priklausomybių medyje aptikta keletas skirtingų
to paties artefakto versijų. Priklausomybių mediacija artefakto versiją pasirenka pagal tai, kuri artefakto
versija priklausomybių medyje yra aukščiausia. Jei tame pačiame priklausomybių medžio lygyje sutinkamos kelios
artefakto versijos, pasirenkama pirmoji paskelbta priklausomybės versija \cite{MAVENd}.

\begin{figure}[H]
    \centering
    \includegraphics[width=\textwidth]{mavenv2}
    \caption{Priklausomybių mediacija}
\end{figure}

\subsubsection{Priklausomybių valdymas}
Maven siūlo ir kitus priklausomybių valdymo mechanizmus. Priklausomybių valdymas (ang. dependency management)
leidžia konkrečiai specifikuoti, kokios priklausomybių versijos bus naudojamos, jei
tos priklausomybės atsidurs tarp projekto tranzityvių priklausomybių. Tai itin paranku, jei priklausomybių
mediavimo (ang. dependency mediation) metu buvo atsiųstos ne tos versijos priklausomybės \cite{MAVENd}.

\subsubsection{Pašalintos bei pasirenkamosios priklausomybės}
Pašalintos priklausomybės (ang. excluded dependencies) suteikia galimybę uždrausti
projektui atsisiųsti tam tikrų nurodytų priklausomybių, net jei tos priklausomybės
priklauso projekto tranzityvioms priklausomybėms. Pasirenkamos priklausomybės (ang. optional dependencies) - leidžia
ignoruoti tam tikras nurodytas tranzityvias priklausomybes iki kol projekto autorius aiškiai nenurodo naudoti
šias pasirenkamas priklausomybes. Pašalintų bei pasirenkamųjų priklausomybių mechanizmai leidžia
vartotojui sumažinti projekto priklausomybių medį, taip pagreitinant projekto programinio kodo surinkimą (ang. build) \cite{MAVENd}.

\subsection{Priklausomybių versijavimas Maven sistemoje}

Maven sistemoje dažniausiai laikomasi priklausomybių semantinio versijavimo modelio -
artefakto versiją sudaro Major Version, Minor Version, Incremental Version
(taip pat gali būti ir Build Number, Qualifier). Toks semantinis versijavimas priklausomybių
naudotojui leidžia susidaryti lūkesčius, kurios priklausomybės versijos yra atgaliai suderinamos,
o kurios - ne. Semantiniame versijavime tik pagrindinės (ang. major) versijos pokyčiai reiškia atgaliai
nesuderinami, todėl vartotojas privalo atsargiai atlikti pagrindinių versijų atnaujinimus. Semantinis versijavimas
leidžia išvengti nestabilaus API problemos, kuri buvo matoma senojoje Go priklausomybių valdymo sistemoje \cite{ORACLEa}.

Maven pom.xml faile taip pat galima nurodyti leidžiamas artefaktų versijų ribas, iš kurių bus pasirenkama tuo metu
didžiausia. Pavyzdžiui, (,  1.0] reiškia, jog artefakto versija bus automatiškai atnaujinama be vartotojo įsikišimo
iki 1.0 versijos, vėliau jau reikės pačio vartotojo įsikišimo. Ši sintaksė buvo labiau naudojama Maven 2, tačiau vis
dar sutinkama ir šiomis dienomis. Galimybė turėti automatinius versijų atnaujinimus reiškia, jog bet kada gali pasikeisti
programinio kodo surinkimo rezultatas ar įvykti klaida. Dėl šios priežasties geriau vengti automatinio versijų atnaujinimo.
Maven 3 taip pat dėl stabilus programinio kodo surinkimo problemų atsisakė LATEST bei RELEASE artefaktų versijų \cite{LIG18}.
%do smth about this

\section{Aptartų priklausomybių valdymo sistemų palygimas}

\subsection{Priklausomybių versijų pasirinkimas}

Kiekviena iš darbe aptartų priklausomybių valdymo sistemų turi skirtingus priklausomybių versijų pasirinkimo metodus.

Pirmoji kursiniame darbe aprašyta Go priklausomybių valdymo sistema renkasi mažiausią leidžiamą priklausomybės versiją -
šis algoritmas vadinamas „minimal version selection“. „Minimal version selection”
prioritetizuojamas projekto priklausomybių versijų stabilumas ir nuspėjamumas, o ne jų naujumas.

Priešingai nei Go, NPM priklausomybių valdymo sistemos strategija yra (neturint package-lock.json failo)
rinktis naujausias leidžiamas priklausomybių versijas. NPM teikia pirmenybę naujausių priklausomybių versijų
naudojimui, net jei tai reiškia, jog sistemos vartotojas gali lengvai gauti priklausomybes su dar nežinomais trikiais.

Maven priklausomybių valdymo sistema siūlo keletą strategijų priklausomybių versijoms nustatyti. Tiesioginėms priklausomybėms,
nurodytoms projekto pom.xml faile, kaip ir NPM sistemoje naudojamas didžiausios leidžiamos versijos algoritmas.
Tranzityvių priklausomybių versijų pasirinkimo numatytasis metodas yra priklausomybių mediacija (ang. dependency mediation),
kuri renkasi pirmąją priklausomybės versiją, esančią priklausomybių medyje. Naudojantis priklausomybių mediacija sunku
nuspėti, kokių versijų tranzityvios priklausomybės bus gautos, šis metodas nėra toks nuspėjamas kaip Go ar NPM sistemų.
Priklausomybių mediacijos metu gali būti gaunamos ir per senų, ir gerokai per naujų versijų priklausomybės, todėl Maven
naudojamas ir priklausomybių valdymo (ang. dependency management) mechanizmas, leidžiantis specifikuoti konkrečias norimų
tranzityvių priklausomybių versijas. Šis metodas nėra patogus, nes reikalauja žinoti konkrečias norimų trazityvių priklausomybių
versijas, tačiau jis leidžia pataisyti priklausomybių mediacijos metu padarytas klaidas. Maven taip pat turi mechanizmų
uždrausti atsiųsti nurodytas tranzityvias priklausomybes ar jas ignoruoti iki kol bus išreikštai nurodoma jas naudoti.

Apibendrinus, Go priklausomybių versijų pasirinkimo algoritmas yra saugus ir nuspėjamas, bet nėra įrašomos pačios
naujausios priklausomybių versijos. NPM sistema leidžia patogiai gauti naujausias priklausomybių versijas, tačiau
paaukojama dalis nuspėjamumo ir saugumo. Maven numatytasis trazityvių priklausomybių mediacijos algoritmas nėra nuspėjamas,
todėl sistema siūlo ir priklausomybių valdymo mechanizmą, kurio metu atsiunčiamos specifikuotų versijų tranzityvios priklausomybės.
Maven, kitaip nei NPM ir Go sistemos, suteikia galimybę vartotojui nenaudoti nurodytų tranzityvių priklausomybių,
taip sumažinamas priklausomybių medis.


\subsection{Stabilių priklausomybių versijų užtikrimas}

Stabilių priklausomybių išlaikymas skirtingose aplinkose yra dar viena svarbi priklausomybių valdymo sistemų
funkcija, kurią kursiniame darbe aptartos sistemos vykdo naudodamos skirtingais metodais.

Go priklausomybių valdymo sistema palaiko stabilias priklausomybes naudodama \enquote{minimal selection} algoritmą,
kurio metu pasirenkamos mažiausių leidžiamų versijų priklausomybės. Šis metodas užtikrina, jog bus gaunamos tos
pačios priklausomybės, nes seniausios leidžiamos priklausomybių versijos nekinta.

NPM pasirenka kitą, daugelyje šiuolaikinių priklausomybių valdymo sistemų populiarų metodą - priklausomybių
rakinimą (ang. dependency locking). Naudojant priklausomybių rakinimą, projekto “lock” faile (package-lock.json) užfiksuojamas
visas projekto priklausomybių medis bei užtikrina, jog instaliuojant priklausomybes bus replikuojamas būtent package-lock.json
užfiksuotas priklausomybių medis.

Maven priklausomybių valdymo sistema neturi tvirtų mechanizmų, leidžiančių užtikrinti priklausomybių stabilumą.
Sistemos vieninteliame manifest faile (pom.xml) leidžiama nurodyti ne tik konkrečias priklausomybių versijas, bet ir
priklausomybių galimų versijų diapazoną, iš kurio naudojama naujausia. Tai reiškia, jog projekto priklausomybių medis,
tiek vystymo, tiek produkcinėje gali būti nestabilus neatsargiam vartotojui nenurodžius konkrečių priklausomybių versijų. Vystant
Maven projektą sunku išlaikyti stabilias priklausomybių versijas ir dėl priklausomybių mediacijos algoritmo, naudojamo gauti
tranzityvioms projekto priklausomybėms, išreikštai nenurodytoms pom.xml failo \enquote{dependencyManagement} mazge.
Priklausomybių mediacija naudoja priklausomybių medyje pirmą rastą priklausomybės versiją, kas reiškia,
jog pridėjus naujas tiesines priklausomybes, gali netikėtai projekto tranzityvių priklausomybių versijos.

Taigi, Go ir NPM turi savitus, tačiau patikimus stabilių priklausomybių užtikrinimo metodus, ir šiuo aspektu yra pranašesni
už Maven. Maven sistemoje nėra saugiklių priklausomybių versijoms užtikrinti (tokių kaip NPM priklausomybių rakinimas), todėl
neatsargūs vartotojai gali susidurti su nestabilių priklausomybių sukeltais trikiais.


\subsection{Priklausomybių versijavimas}

Visos iš pasirinktų priklausomybių valdymo sistemų priklausomybių versijų sudarymą grindžia semantiniu versijavimu.

NPM sistema naudoja tradicinį semantinį versijavimą, kuomet priklausomybės versija nurodoma trimis skaičiais x.y.z
(x - pagrindinė (ang. major) versija, y - antraeilė (ang. minor) versija, z - pataisymų (ang. patch) versija).
Ši sistema intuityvi vartotojui bei naudojama daugelyje modernių priklausomybių valdymo sistemų tokių kaip PyPi, dep.

Go sistema naudoja semantinį importų versijavimą (ang. semantic import versioning). Semantiniame importų versijavime pagrindinė
(ang. major versija) yra nurodyma pačiame priklausomybės importavimo kelyje (pavyzdžiui,
\enquote{github.com/greta/foo/v2}), tuo tarpu antraeilė bei pataisymų versijos nurodomos įprastu būdu.
Šis Go sistemos metodas lemia, jog vartotojai privalo būti labiau dėmesingi atnaujindami priklausomybes, nes
norint pakeisti pagrindinę naudojamos priklausomybės versiją reikia naudoti kitą priklausomybės importavimo kelią.

Maven remiasi tradiciniu semantiniu versijavimu, tačiau priklausomybės versijoje taip pat gali būti nurodoma ir
Build Number bei Qualifier. Build Number leidžia vartotojui specifikuoti konkretų priklausomybės build, o naudojant
Qualifier galima išbandyti ir testuoti dar oficialiai neišleistas priklausomybių versijas, pavyzdžiui, -Alpha Qualifier
reiškia, jog priklausomybė praėjo tik baltos dėžės testavimą. Galimybė naudoti dar vystomas priklausomybių versijas yra patogus
būdas testuoti priklausomybių kokybę ir pranešti apie trikius, tačiau tai darydamas vartotojas turi būti atidus ir
nenaudoti šių versijų produkcinėje aplinkoje, nes šios priklausomybės nėra stabilios.

Apibendrinus, visos šios sistemos remiasi semantiniu versijavimu, tačiau taiko jį skirtingai, prioritetizuodamos skirtingus priklausomybių
valdymo aspektus. NPM naudojama gerai žinoma tradicinė semantinio versijavimo sistema, kuri yra intuityvi vartotojams ir paprasta
naudoti. Go sistema priklausomybės pagrindinės versijos numerį specifikuoja priklausomybės importavimo kelyje, kas lemia, jog priklausomybių
atnaujinimas reikalauja daugiau atidumo nei NPM ar Maven sistemose. Tai yra pozityvus pokytis, nes neatsargus pagrindinės versijos atnaujinimas
gali lengvai sukurti trikių vystomoje programų sistemoje. Maven sistema yra pranaši tuo, jog įgalina vartotoją patogiai testuoti
vystymo fazėje esančias priklausomybes, tačiau tai kelia grėsmę, jog nestabilios priklausomybės neatsargaus vartotojo
bus naudojamos ir produkcinėje aplinkoje.


\subsection{Aptartų priklausomybių valdymo sistemų apibendrinimas}

%[Gal būtų galima pridėti tekstą šiame skyriuje apibendrinančią lentelę? T.y. apibendrintą glaustą, palyginimą]

\begin{longtable}{| p{.20\textwidth} | p{.20\textwidth} | p{.20\textwidth} | p{.30\textwidth}|}
    \caption{Aptartų priklausomybių valdymo sistemų savybių palyginimas} \label{tab:expirement-table-results}\\ \hline
     & \textbf{Go} & \textbf{NPM} & \textbf{Maven}  \\ \hline

    %priklausomybiu versiju parinkimas
    \textbf{Priklausomybių versijų parinkimas}
    & Naudojama seniausia leidžiama priklausomybės versija.
    & Prieš sukuriant package-lock.json failą, renkamasi naujausia leidžiama versija, vėliau renkamasi pagal package-lock.json faile nurodytas versijas.
    & Priklausomybių valdymo mechanizmas parenka naujausią versiją iš vartotojo nurodytų leistinų priklausomybės versijų, priklausomybių mediacija
    renkasi aukščiausiai priklausomybių medyje esančią priklausomybę. \\ \hline

    %Stabilių priklausomybių versijų užtikrinimas
    \textbf{Stabilių priklausomybių versijų užtikrinimas}
    & Naudojama seniausia leidžiama priklausomybės versija.
    & Priklausomybių rakinimas naudojant package-lock.json failą.
    & Nėra tvirtų mechanizmų leidžiančių užtikrinti stabilias versijas. \\ \hline

    %versijavimas
    \textbf{Priklausomybių versijavimas}
    & Semantinis importų versijavimas.
    & Semantinis versijavimas.
    & Semantinis versijavimas, papildomai galima pridėti Build Number bei Qualifier. \\ \hline
\end{longtable}