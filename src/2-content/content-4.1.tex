\subsection{Priklausomybių versijų pasirinkimas}

Kiekviena iš darbe aptartų priklausomybių valdymo sistemų turi skirtingus priklausomybių versijų pasirinkimo metodus.

Pirmoji kursiniame darbe aprašyta Go priklausomybių valdymo sistema renkasi mažiausią leidžiamą priklausomybės versiją -
šis algoritmas vadinamas „minimal version selection“. „Minimal version selection”
prioritetizuojamas projekto priklausomybių versijų stabilumas ir nuspėjamumas, o ne jų naujumas.

Priešingai nei Go, NPM priklausomybių valdymo sistemos strategija yra (neturint package-lock.json failo)
rinktis naujausias leidžiamas priklausomybių versijas. NPM teikia pirmenybę naujausių priklausomybių versijų
naudojimui, net jei tai reiškia, jog sistemos vartotojas gali lengvai gauti priklausomybes su dar nežinomais trikiais.

Maven priklausomybių valdymo sistema siūlo keletą strategijų priklausomybių versijoms nustatyti. Tiesioginėms priklausomybėms,
nurodytoms projekto pom.xml faile, kaip ir NPM sistemoje naudojamas didžiausios leidžiamos versijos algoritmas.
Tranzityvių priklausomybių versijų pasirinkimo numatytasis metodas yra priklausomybių mediacija (ang. dependency mediation),
kuri renkasi pirmąją priklausomybės versiją, esančią priklausomybių medyje. Naudojantis priklausomybių mediacija sunku
nuspėti, kokių versijų tranzityvios priklausomybės bus gautos, šis metodas nėra toks nuspėjamas kaip Go ar NPM sistemų.
Priklausomybių mediacijos metu gali būti gaunamos ir per senų, ir gerokai per naujų versijų priklausomybės, todėl Maven
naudojamas ir priklausomybių valdymo (ang. dependency management) mechanizmas, leidžiantis specifikuoti konkrečias norimų
tranzityvių priklausomybių versijas. Šis metodas nėra patogus, nes reikalauja žinoti konkrečias norimų trazityvių priklausomybių
versijas, tačiau jis leidžia pataisyti priklausomybių mediacijos metu padarytas klaidas. Maven taip pat turi mechanizmų
uždrausti atsiųsti nurodytas tranzityvias priklausomybes ar jas ignoruoti iki kol bus išreikštai nurodoma jas naudoti.

Apibendrinus, Go priklausomybių versijų pasirinkimo algoritmas yra saugus ir nuspėjamas, bet nėra įrašomos pačios
naujausios priklausomybių versijos. NPM sistema leidžia patogiai gauti naujausias priklausomybių versijas, tačiau
paaukojama dalis nuspėjamumo ir saugumo. Maven numatytasis trazityvių priklausomybių mediacijos algoritmas nėra nuspėjamas,
todėl sistema siūlo ir priklausomybių valdymo mechanizmą, kurio metu atsiunčiamos specifikuotų versijų tranzityvios priklausomybės.
Maven, kitaip nei NPM ir Go sistemos, suteikia galimybę vartotojui nenaudoti nurodytų tranzityvių priklausomybių,
taip sumažinamas priklausomybių medis.
