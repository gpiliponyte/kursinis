\subsection{Priklausomybių versijavimas NPM sistemoje}

NPM priklausomybėms versijuoti naudojama semantinio versijavimo sistema. Ši sistema versijuojamui vienetui
suteikia X.Y.Z formos versiją, kurioje X reiškia pagrindinę (ang. major) versiją, Y - antraeilę (ang. minor) versiją,
Z - pataisymų (ang. patch) versiją. Pataisymų (ang. patch) ir antraeilių (ang. minor) versijų pasikeitimai yra atgaliai suderinami
ir yra saugūs naudoti projektuose su ta pačia pagrindine (ang. major) versija \textsuperscript{[NPMb]}.
Tuo tarpu pagrindinės versijos pasikeitimas reiškia atgaliai nesuderinamus pokyčius.
Semantinis versijavimas leidžia vartotojui susidaryti lūkesčius naujoms priklausomybių versijoms bei
package.json faile apibrėžti kurios priklausomybių versijos ir kurie jų atnaujinimai yra leidžiami projekte.
Ši versijavimo sistema išsprendžia nestabilaus API problemą. %elaborate gurl
