\subsection{Priklausomybių versijavimas}

Visos iš pasirinktų priklausomybių valdymo sistemų priklausomybių versijų sudarymą grindžia semantiniu versijavimu.

NPM sistema naudoja tradicinį semantinį versijavimą, kuomet priklausomybės versija nurodoma trimis skaičiais x.y.z
(x - pagrindinė (ang. major) versija, y - antraeilė (ang. minor) versija, z - pataisymų (ang. patch) versija).
Ši sistema intuityvi vartotojui bei naudojama daugelyje modernių priklausomybių valdymo sistemų tokių kaip PyPi, dep.

Go sistema naudoja semantinį importų versijavimą (ang. semantic import versioning). Semantiniame importų versijavime pagrindinė
(ang. major versija) yra nurodyma pačiame priklausomybės importavimo kelyje (pavyzdžiui,
\enquote{github.com/greta/foo/v2}), tuo tarpu antraeilė bei pataisymų versijos nurodomos įprastu būdu.
Šis Go sistemos metodas lemia, jog vartotojai privalo būti labiau dėmesingi atnaujindami priklausomybes, nes
norint pakeisti pagrindinę naudojamos priklausomybės versiją reikia naudoti kitą priklausomybės importavimo kelią.

Maven remiasi tradiciniu semantiniu versijavimu, tačiau priklausomybės versijoje taip pat gali būti nurodoma ir
Build Number bei Qualifier. Build Number leidžia vartotojui specifikuoti konkretų priklausomybės build, o naudojant
Qualifier galima išbandyti ir testuoti dar oficialiai neišleistas priklausomybių versijas, pavyzdžiui, -Alpha Qualifier
reiškia, jog priklausomybė praėjo tik baltos dėžės testavimą. Galimybė naudoti dar vystomas priklausomybių versijas yra patogus
būdas testuoti priklausomybių kokybę ir pranešti apie trikius, tačiau tai darydamas vartotojas turi būti atidus ir
nenaudoti šių versijų produkcinėje aplinkoje, nes šios priklausomybės nėra stabilios.

Apibendrinus, visos šios sistemos remiasi semantiniu versijavimu, tačiau taiko jį skirtingai, prioritetizuodamos skirtingus priklausomybių
valdymo aspektus. NPM naudojama gerai žinoma tradicinė semantinio versijavimo sistema, kuri yra intuityvi vartotojams ir paprasta
naudoti. Go sistema priklausomybės pagrindinės versijos numerį specifikuoja priklausomybės importavimo kelyje, kas lemia, jog priklausomybių
atnaujinimas reikalauja daugiau atidumo nei NPM ar Maven sistemose. Tai yra pozityvus pokytis, nes neatsargus pagrindinės versijos atnaujinimas
gali lengvai sukurti trikių vystomoje programų sistemoje. Maven sistema yra pranaši tuo, jog įgalina vartotoją patogiai testuoti
vystymo fazėje esančias priklausomybes, tačiau tai kelia grėsmę, jog nestabilios priklausomybės neatsargaus vartotojo
bus naudojamos ir produkcinėje aplinkoje.
