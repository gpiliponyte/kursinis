\subsection{Stabilių priklausomybių versijų užtikrimas}

Stabilių priklausomybių išlaikymas skirtingose aplinkose yra dar viena svarbi priklausomybių valdymo sistemų
funkcija, kurią kursiniame darbe aptartos sistemos vykdo naudodamos skirtingais metodais.

Go priklausomybių valdymo sistema palaiko stabilias priklausomybes naudodama \enquote{minimal selection} algoritmą,
kurio metu pasirenkamos mažiausių leidžiamų versijų priklausomybės. Šis metodas užtikrina, jog bus gaunamos tos
pačios priklausomybės, nes seniausios leidžiamos priklausomybių versijos nekinta.

NPM pasirenka kitą, daugelyje šiuolaikinių priklausomybių valdymo sistemų populiarų metodą - priklausomybių
rakinimą (ang. dependency locking). Naudojant priklausomybių rakinimą, projekto “lock” faile (package-lock.json) užfiksuojamas
visas projekto priklausomybių medis bei užtikrina, jog instaliuojant priklausomybes bus replikuojamas būtent package-lock.json
užfiksuotas priklausomybių medis.

Maven priklausomybių valdymo sistema neturi tvirtų mechanizmų, leidžiančių užtikrinti priklausomybių stabilumą.
Sistemos vieninteliame manifest faile (pom.xml) leidžiama nurodyti ne tik konkrečias priklausomybių versijas, bet ir
priklausomybių galimų versijų diapazoną, iš kurio naudojama naujausia. Tai reiškia, jog projekto priklausomybių medis,
tiek vystymo, tiek produkcinėje gali būti nestabilus neatsargiam vartotojui nenurodžius konkrečių priklausomybių versijų. Vystant
Maven projektą sunku išlaikyti stabilias priklausomybių versijas ir dėl priklausomybių mediacijos algoritmo, naudojamo gauti
tranzityvioms projekto priklausomybėms, išreikštai nenurodytoms pom.xml failo \enquote{dependencyManagement} mazge.
Priklausomybių mediacija naudoja priklausomybių medyje pirmą rastą priklausomybės versiją, kas reiškia,
jog pridėjus naujas tiesines priklausomybes, gali netikėtai projekto tranzityvių priklausomybių versijos.

Taigi, Go ir NPM turi savitus, tačiau patikimus stabilių priklausomybių užtikrinimo metodus, ir šiuo aspektu yra pranašesni
už Maven. Maven sistemoje nėra saugiklių priklausomybių versijoms užtikrinti (tokių kaip NPM priklausomybių rakinimas), todėl
neatsargūs vartotojai gali susidurti su nestabilių priklausomybių sukeltais trikiais.
