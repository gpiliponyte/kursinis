\subsection{Priklausomybių valdymo sistemų poreikis}

Programų sistemos dažnai susideda iš mažesnių, vieną paskirtį turinčių (ang. single purpose) modulių.
Populiarėjant atvirojo programinio kodo (ang. open source) naudojimui, sistemos neretai priklauso nuo išorinių,
kitų autorių sukurtų modulių \cite{PAD17}. Toks programinio kodo daugkartinis panaudojimas turi daug privalumų - paspartinamas programų
sistemų kūrimo procesas, daugiau laiko skiriama fokusuojantis į unikalias kuriamos sistemos užduotis.

Priklausomybių valdymo sistemos yra įrankiai, palengvinantys išorinių modulių naudojimą programų sistemose.
Šios sistemos leidžia atsisiųsti tiesiogines projekto priklausomybes, dauguma jų taip pat turi mechanizmus gauti
ir projekto tranzityvias priklausomybes (tiesioginių priklausomybių priklausomybes) \cite{PAD17}. Naudojant
priklausomybių valdymo sistemas vartotojui būtina nurodyti tik norimų tiesioginių priklausomybių sąrašą,
tinkamos tranzityvių priklausomybių versijos bei iš kokių repozitorijos jos bus siunčiamos nustatoma
automatiškai. Priklausomybių valdymo sistemos taip pat dažnai suteikia galimybę patogiai atnaujinti individualias
priklausomybes arba visą projekto priklausomybių medį \cite{MAT17}.

Priklausomybių valdymo sistemos yra svarbios kuriant programų sistemas komandoje bei kuriamos programų
sistemos nuolatinės integracijos bei nuolatinio tiekimo (ang. continuous integration and continuous delivery,
trump. CI/CD) procesui. Modernios priklausomybių valdymo sistemos leidžia užtikrinti, jog visi komandos
inžinieriai dirba su tų pačių versijų priklausomybėmis. Šios sistemos taip pat išlaiko vienodas priklausomybes
tiek vystymo, tiek testinėse bei produkcinėje aplinkose \cite{MAT17}. Taip sumažinama su netinkamomis
priklausomybėmis susijusių programų sistemos trikių tikimybė.

