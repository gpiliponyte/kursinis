\subsection{Priklausomybių versijavimas Maven sistemoje}

Maven sistemoje dažniausiai laikomasi priklausomybių semantinio versijavimo modelio -
artefakto versiją sudaro Major Version, Minor Version, Incremental Version
(taip pat gali būti nurodoma ir Build Number, Qualifier). Naudojant semantinį versijavimą priklausomybių naudotojui
lengviau susidaryti lūkesčius, kurios priklausomybės versijos yra atgaliai suderinamos, o kurios - ne. Semantiniame versijavime
tik pagrindinės (ang. major) versijos pokyčiai reiškia atgaliai nesuderinami, todėl vartotojas privalo atsargiai
atlikti pagrindinių versijų atnaujinimus. Semantinis versijavimas leidžia išvengti nestabilaus API problemos, kuri buvo matoma
senojoje Go priklausomybių valdymo sistemoje \cite{ORACLEa}.

Maven versijose kartais naudojami BuildNumber bei Qualifier vartotojams suteikia dar daugiau priklausomybių valdymo galimybių.
Nurodant BuildNumber galima specifikuoti, kokį artifakto programinio kodo surinkimą (ang. build) norima naudoti, taip palengvinamas
programų sistemos vystymo procesas \cite{ORACLEa}. Qualifier identifikatorius leidžia išbandyti dar vystymo stadijoje
esančias priklausomybių versijas, pavyzdžiui, Alpha Qualifier nurodo, jog priklausomybės versija yra praėjusi tik baltos
dėžės testavimą. Naudojant Qualifier patogu iš anksto testuoti vystomas priklausomybes bei pranešti apie rastus trikius.
Vartotojas turėtų būti atidus nenaudoti priklausomybių versijų su Qualifier produkcinėse aplinkose, nes šios priklausomybės nėra
stabilios ir gali turėti klaidų \cite{VB14}.

Maven pom.xml faile galima nurodyti ir leidžiamas artefaktų versijų ribas, iš kurių bus pasirenkama tuo metu didžiausia.
Pavyzdžiui, nurodant (,  1.0] kaip artifakto versiją, artifakto versija bus atnaujinama be vartotojo įsikišimo iki 1.0 versijos,
vėliau reikės pačio vartotojo įsikišimo. Ši sintaksė buvo plačiau naudojama Maven 2, tačiau vis dar sutinkama ir šiomis dienomis.
Galimybė turėti automatinius versijų atnaujinimus reiškia, jog bet kada gali pasikeisti programinio kodo surinkimo rezultatas ar
įvykti klaida. Dėl šios priežasties geriau vengti automatinio versijų atnaujinimo. Maven 3 taip pat dėl stabilus programinio kodo
surinkimo problemų atsisakė LATEST bei RELEASE artefaktų versijų \cite{LIG18}.
