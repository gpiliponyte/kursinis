\subsection{Priklausomybių versijavimas Maven sistemoje}

Maven sistemoje dažniausiai laikomasi priklausomybių semantinio versijavimo modelio -
artefakto versiją sudaro Major Version, Minor Version, Incremental Version
(taip pat gali būti ir Build Number, Qualifier). Toks semantinis versijavimas priklausomybių
naudotojui leidžia susidaryti lūkesčius, kurios priklausomybės versijos yra atgaliai suderinamos,
o kurios - ne. Semantiniame versijavime tik pagrindinės (ang. major) versijos pokyčiai reiškia atgaliai
nesuderinami, todėl vartotojas privalo atsargiai atlikti pagrindinių versijų atnaujinimus. Semantinis versijavimas
leidžia išvengti nestabilaus API problemos, kuri buvo matoma senojoje Go priklausomybių valdymo sistemoje \cite{ORACLEa}.

Maven pom.xml faile taip pat galima nurodyti leidžiamas artefaktų versijų ribas, iš kurių bus pasirenkama tuo metu
didžiausia. Pavyzdžiui, (,  1.0] reiškia, jog artefakto versija bus automatiškai atnaujinama be vartotojo įsikišimo
iki 1.0 versijos, vėliau jau reikės pačio vartotojo įsikišimo. Ši sintaksė buvo labiau naudojama Maven 2, tačiau vis
dar sutinkama ir šiomis dienomis. Galimybė turėti automatinius versijų atnaujinimus reiškia, jog bet kada gali pasikeisti
programinio kodo surinkimo rezultatas ar įvykti klaida. Dėl šios priežasties geriau vengti automatinio versijų atnaujinimo.
Maven 3 taip pat dėl stabilus programinio kodo surinkimo problemų atsisakė LATEST bei RELEASE artefaktų versijų \cite{LIG18}.
%do smth about this