\subsection{Priklausomybių versijų pasirinkimas NPM sistemoje}

Norint naudotis NPM priklausomybėmis, projekte būtina turėti package.json failą. Šį failą
sudaro projekto meta-data, tarp kurios nurodomos ir reikiamos priklausomybės bei jų versijos. Package.json
faile išskirtos dvi priklausomybių grupės - \enquote{dependencies} (projekto priklausomybės) bei \enquote{devDependencies}
(projekto kūrimui ir testavimui reikalingos priklausomybės). Kiekvienai šiame faile nurodytai priklausomybei nurodoma ir jos versija \textsuperscript{[GAU18]}.
NPM leidžia nurodyti ne tik tikslias priklausomybių versijas, bet ir leidžiamų versijų diapazoną.
Tildės ženklas (~) leidžia naujesnes pataisymų (ang. patch) versijas,
stogelio ženklas (\textasciicircum) leidžia naujesnes antraeiles (ang. minor) arba pataisymų (ang. patch) versijas \textsuperscript{[PIT15]}.


Galimybė package.json faile esančioms priklausomybėms suteikti leistinų versijų diapazoną turi privalumų ir trūkumų.
Šis metodas leidžia vartotojui patogiau atnaujinti turimas priklausomybes - įvykdžius
\enquote{npm update} komandą automatiškai gaunamos naujausios leistinos priklausomybių versijos vartotojui net
nekeitus package.json \textsuperscript{[PIT15]}. Šio metodo trūkumas yra, jog nurodant leistinų priklausomybių versijų ribas,
dirbant komandoje nėra galimybės užtikrinti, jog įvykdžius \enquote{npm install} su tuo pačiu package.json bus gautos
tų pačių versijų priklausomybės \textsuperscript{[PIT15]}. Išlaikyti stabilias priklausomybių versijas projekte yra svarbu norint
užtikrinti stabilų programos surinkimą (ang. stable build) - jei naujausioje priklausomybės versijoje būtų
klaida, projektas veiktų klaidingai arba visai neveiktų.

Norint išspręsti nestabilaus programos surinkimo problemą, NPM 5 versijoje pradedamas naudoti
package-lock.json failas. Šis failas nurodo tikslų priklausomybių medį - visų naudojamų priklausomybių,
bei šių priklausomybių priklausomybių tikslias versijas \textsuperscript{[NPMc]}. Package-lock.json generuojamas automatiškai pakeitus priklausomybių medį
arba package.json failą \textsuperscript{[NPMc]}. Atsiradus package-lock.json failui, NPM sistema vadovaujasi šiuo failu siųsdama priklausomybes.
Kadangi package-lock.json įrašomos tikslios priklausomybių versijos, išsprendžiama problema,
jog kartu dirbantiems kolegoms ar CI/CD serveriui atsiunčiamos netinkamų versijų priklausomybės.
