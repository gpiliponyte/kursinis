\subsection{Aptartų priklausomybių valdymo sistemų apibendrinimas}

%[Gal būtų galima pridėti tekstą šiame skyriuje apibendrinančią lentelę? T.y. apibendrintą glaustą, palyginimą]

\begin{longtable}{| p{.20\textwidth} | p{.20\textwidth} | p{.20\textwidth} | p{.30\textwidth}|}
    \caption{Aptartų priklausomybių valdymo sistemų savybių palyginimas} \label{tab:expirement-table-results}\\ \hline
     & \textbf{Go} & \textbf{NPM} & \textbf{Maven}  \\ \hline

    %priklausomybiu versiju parinkimas
    \textbf{Priklausomybių versijų parinkimas}
    & Naudojama seniausia leidžiama priklausomybės versija.
    & Prieš sukuriant package-lock.json failą, renkamasi naujausia leidžiama versija, vėliau renkamasi pagal package-lock.json faile nurodytas versijas.
    & Priklausomybių valdymo mechanizmas parenka naujausią versiją iš vartotojo nurodytų leistinų priklausomybės versijų, priklausomybių mediacija
    renkasi aukščiausiai priklausomybių medyje esančią priklausomybę. \\ \hline

    %Stabilių priklausomybių versijų užtikrinimas
    \textbf{Stabilių priklausomybių versijų užtikrinimas}
    & Naudojama seniausia leidžiama priklausomybės versija.
    & Priklausomybių rakinimas naudojant package-lock.json failą.
    & Nėra tvirtų mechanizmų leidžiančių užtikrinti stabilias versijas. \\ \hline

    %versijavimas
    \textbf{Priklausomybių versijavimas}
    & Semantinis importų versijavimas.
    & Semantinis versijavimas.
    & Semantinis versijavimas, papildomai galima pridėti Build Number bei Qualifier. \\ \hline
\end{longtable}